\documentclass[aspectratio=169,xcolor=dvipsnames]{beamer}
\usetheme{SimplePlus} 

\usepackage[utf8]{inputenc}
\usepackage[T1]{fontenc}
\usepackage{lmodern}
\usepackage{graphicx}
\usepackage{booktabs}
\usepackage{amsmath,amssymb} 
\usepackage{hyperref}

\title{Algoritmi de Reprezentare Rară, Învățarea Dicționarelor şi Reconstrucție Rară\\
\large Detectarea și Clasificarea Aritmiilor EKG pe 12 Derivații}
\author{Tudor Pistol și Teofil Simiraș}
\date{\today}

\begin{document}

%----------------------------------
\begin{frame}
  \titlepage
\end{frame}
%----------------------------------

\begin{frame}{Cuprins}
  \tableofcontents
\end{frame}

%=============================================================================
\section{Introducere și Context}
%=============================================================================

\begin{frame}{Introducere și Context}
  \begin{itemize}
    \item Semnalele EKG de 12 derivații sunt voluminoase și complexe.
    \item \textbf{Aritmiile cardiace} pot fi critice dacă nu sunt detectate rapid.
    \item Proiectul nostru: 
    \begin{itemize}
      \item Preprocesare (filtrare, segmentare) \\
      \item \textbf{Sparse Coding} (ex. OMP + K-SVD) \\
      \item Clasificare (SVM sau Logistic Regression) 
    \end{itemize}
  \end{itemize}
\end{frame}

%=============================================================================
\section{Problema Abordată}
%=============================================================================

\begin{frame}{Problema Abordată}
  \begin{block}{Obiectiv principal}
    \textbf{Detectarea și clasificarea aritmiilor cardiace} pe EKG de 12 derivații prin \emph{algoritmi de reprezentare rară} și \emph{învățarea dicționarelor}.
  \end{block}
  \begin{alertblock}{De ce e important?}
    \begin{itemize}
      \item Aritmiile cardiace sunt frecvente și pot deveni periculoase.
      \item Sistem automat: reduce timpul de analiză și crește acuratețea (spitale, dispozitive portabile).
    \end{itemize}
  \end{alertblock}
\end{frame}

%=============================================================================
\section{Preprocesarea Semnalelor EKG}
%=============================================================================

\begin{frame}{Filtrare Butterworth / Chebyshev}
  \begin{itemize}
    \item Folosim un filtru \textbf{band-pass} (0.5 -- 40 Hz) pentru a reține componentele relevante din EKG.
    \item \textbf{Butterworth}: tranziție lină, fără ripple în banda de trecere.
    \item \textbf{Chebyshev}: roll-off mai abrupt, dar introduce ripple.
  \end{itemize}
\end{frame}

\begin{frame}{Segmentare \& Normalizare}
  \begin{itemize}
    \item \textbf{Segmentare}: Detectăm complexul QRS (ex. algoritm Pan-Tompkins).
    \item Extragem fereastră (ex. 100 ms înainte, 300 ms după R-peak).
    \item \textbf{Normalizare}: Scalam amplitudinile în intervalul [-1, 1].
  \end{itemize}
\end{frame}

%=============================================================================
\section{Reprezentarea Rară (Sparse Coding)}
%=============================================================================

\begin{frame}{Definiție Sparse Coding}
  \textbf{Reprezentare rară} a unui semnal:
  \[
    y \in \mathbb{R}^m,\quad 
    D \in \mathbb{R}^{m \times n}, \quad
    y = D\,x,
  \]
  unde \(\| x \|_0\) (\# de elemente nenule) este mic. 

  \begin{itemize}
    \item Dicționarul \(D\) conține \emph{atomi} (coloane).
    \item Doar câțiva atomi (coeficienți) sunt nenuli în \(x\).
  \end{itemize}
\end{frame}

\begin{frame}{Probleme de optimizare}
  \textbf{1. Minimizare a erorii sub constrângere de raritate}:
  \[
    \min_{x} \|y - D\,x\|^2 
    \quad \text{s.t.} \quad 
    \|x\|_0 \le s.
  \]
  \textbf{2. Minimizare a rarității sub constrângere de eroare}:
  \[
    \min_{x} \|x\|_0
    \quad \text{s.t.} \quad
    \|y - D\,x\| \le \varepsilon.
  \]
\end{frame}

%=============================================================================
\section{Algoritmi Sparse (OMP și K-SVD)}
%=============================================================================

\begin{frame}{OMP (Orthogonal Matching Pursuit)}
  \begin{itemize}
    \item Algoritm \emph{greedy} pentru a determina suportul vectorului rar.
    \item \(\textbf{Pași principali}:\)
    \begin{enumerate}
      \item Reziduul inițial \(e = y\), suportul \(S = \emptyset\).
      \item \(\displaystyle k = \arg \max_{j \notin S} |e^T d_j|\).
      \item \(S = S \cup \{k\}\); rezolvăm \(\min_{x_S} \| y - D_S x_S \|\).
      \item Reziduu: \( e \leftarrow y - D_S x_S\).
      \item Repetăm până \(\|e\|\) mic sau \(|S| = s\).
    \end{enumerate}
  \end{itemize}
\end{frame}

\begin{frame}{Antrenarea Dicționarului - K-SVD}
  \begin{itemize}
    \item Obiectiv: 
    \[
      \min_{D, X} \|Y - D\,X\|_F^2
      \quad\text{s.t.}\quad
      \|x_i\|_0 \le s.
    \]
    \item Alternăm:
    \begin{enumerate}
      \item \emph{Sparse coding} (OMP pentru fiecare coloană \(y_i\)).
      \item \emph{Actualizare dicționar}:
        \begin{itemize}
          \item Calculăm reziduu \(F = Y - D\,X\).
          \item Pentru fiecare atom \(d_j\), se aplică SVD pe submatricea reziduală relevantă pentru a-l optimiza.
        \end{itemize}
    \end{enumerate}
  \end{itemize}
\end{frame}

%=============================================================================
\section{Clasificarea Bătăilor EKG}
%=============================================================================

\begin{frame}{Clasificare EKG (Normal vs. Aritmie)}
  \begin{itemize}
    \item După ce fiecare bătaie EKG este reprezentată de un vector rar \(x\), îl folosim drept \emph{feature vector}.
    \item \textbf{Algoritm de clasificare}:
    \begin{enumerate}
      \item \textbf{SVM} (kernel RBF): foarte folosit în probleme binare (normal vs. aritmie).
      \item \textbf{Regresie Logistică}: mai simplu, interpretabil.
      \item (Opțional) Random Forest, XGBoost, Rețele Neurale etc.
    \end{enumerate}
    \item Se antrenează pe \texttt{train}, se testează pe \texttt{test}; metrici: Acuratețe, Recall, F1, ROC/AUC.
  \end{itemize}
\end{frame}

%=============================================================================
\section{Rezultate și Metodologie de Evaluare}
%=============================================================================

\begin{frame}{Rezultate Așteptate}
  \begin{itemize}
    \item Acuratețe > \textbf{90\%} în detectarea aritmiilor cardiace.
    \item Reducerea zgomotului prin \emph{sparse reconstruction}.
    \item Coeficienții rari \(\implies\) separare mai bună între bătăile normale și anormale.
  \end{itemize}
\end{frame}

\begin{frame}{Metodologie de Evaluare}
  \begin{itemize}
    \item Împărțire \textbf{train / validation / test}.
    \item \textbf{Metrici}: 
    \begin{itemize}
      \item Acuratețe, Sensibilitate (Recall), Specificitate, F1, ROC/AUC.
    \end{itemize}
    \item Comparație cu:
      \begin{itemize}
        \item Fără sparse coding (features brute).
        \item Dicționar random vs. K-SVD.
      \end{itemize}
  \end{itemize}
\end{frame}

%=============================================================================
\section{Concluzii și Direcții}
%=============================================================================

\begin{frame}{Concluzii}
  \begin{itemize}
    \item \textbf{Sparse Coding} (OMP + K-SVD) oferă un mod eficient de extragere a trăsăturilor EKG.
    \item \textbf{Clasificare} (SVM) a aritmiilor cu acuratețe ridicată.
    \item \textbf{Robustețe} la zgomot, potențial de compresie.
  \end{itemize}
\end{frame}

\begin{frame}{Direcții Viitoare}
  \begin{itemize}
    \item \textbf{Detecție multi-clasă}: fibrilație atrială, flutter, blocuri AV etc.
    \item \textbf{Implementare embedded}: pe un dispozitiv Holter, monitorizare real-time.
    \item \textbf{Sparse Autoencoder}: antrenare neuronală cu constrângeri de raritate.
  \end{itemize}
\end{frame}

\begin{frame}{Bibliografie}
  \footnotesize
  \begin{itemize}
    \item \textbf{[1]} \emph{A large scale 12-lead electrocardiogram database for arrhythmia study}, \url{https://physionet.org/content/ecg-arrhythmia/1.0.0/}
    \item \textbf{[2]} \emph{Heart Arrhythmias}, \url{https://www.physio-pedia.com/Heart_Arrhythmias}
    \item \textbf{[3]} \emph{Cursul de Procesarea Semnalelor}, \url{https://cs.unibuc.ro/~crusu/ps/index.html}
    \item \textbf{[4]} \emph{Cursul de Calcul Numeric}, \url{https://numeric.cs.unibuc.ro/cni.html}
  \end{itemize}
\end{frame}

%=============================================================================
\begin{frame}
  \Huge{\centering \textbf{Vă mulțumim!}}
\end{frame}

\end{document}

